\documentclass{beamer}
\usepackage{listings}

\usetheme{progressbar}
\progressbaroptions{titlepage=normal, frametitle=normal}

\title{OBS - OpenSuSE Build Service}
\institute{Elektrobit Wireless(2010)}
\author{Lifu Zhang}
\date{}
\logo{\includegraphics[width=1cm]{linuxfb-logo.png}}

\begin{document}
\setbeamertemplate{background}{\includegraphics[height=\paperheight]{progressbar/slide.png}}

\lstset{% general command to set parameter(s)
basicstyle=\scriptsize,
% print whole listing small
keywordstyle=\color{black}\bfseries\underbar,
% underlined bold black keywords
identifierstyle=,
% nothing happens
stringstyle=\color{purple}\ttfamily,
% typewriter type for strings
showstringspaces=false}
% no special string spaces

\maketitle

\section*{Outline}
\begin{frame}
  \frametitle{Outline}
  \tableofcontents
\end{frame}

\section{Introduciton}
\begin{frame}
  \frametitle{OBS's Ability}
  \begin{itemize}
    \item What OBS Can Do:
      \begin{itemize}
	\item Build RPM Packages For different platforms with smae copy of code.
	\item Publish rpm as 'repository', easy for package mgmt tools to use
      \end{itemize}
    \item What OBS Can NOT:
      \begin{itemize}
	\item Create MeeGo Image.
	\item Testing.
      \end{itemize}
  \end{itemize}
\end{frame}

\begin{frame}
  \frametitle{Terms in OBS: Project, Repository, Arch, Package}
  \begin{itemize}
    \item Package: A software package, contains source tarball, spec file, and (maybe) patches.
    \item Project: A group of Package, with config for build environmet, repositories, and archs.
    \item Repository: A place to organize and publish binary/source packages
    \item Arch: Architectures, a repo may contains multipule archs
  \end{itemize}
\end{frame}

\begin{frame}
  \frametitle{OBS's Interfaces}
  \begin{itemize}
    \item OBS API, a RESTful web interface.
    \item OSC, Command line interface based on API
    \item WebUI, Web application based on OBS API
    \item Event, obs have a way to generate events
  \end{itemize}
\end{frame}

\section{Using OBS}
\subsection{Getting Started}
\begin{frame}
  \frametitle{Hello obs - build an Qt Application for MeeGo}
  \begin{itemize}
    \item Prepare an source tarball
    \item Create .spec file
    \item Create "package" on OBS Server
    \item Upload sources
    \item Monitor Build Process
    \item Get Binary Packages(Build results)
  \end{itemize}
\end{frame}

\subsection{OBS Commandline Client}
\begin{frame}[fragile]
  \frametitle{Setup OSC tool}
  \begin{itemize}
    \item Install osc \\
      Get osc from site http://software.opensuse.org/search/
    \item Script 'oscl' \\
      To operate local obs server, we could use a script 'oscl' 
  \end{itemize}
  \lstset{language=bash}
  \begin{lstlisting}
  #!/bin/bash
  API_URL="http://cnbjas56.cn.ebgroup.elektrobit.com:81"
  export http_proxy="" 
  osc -A $API_URL $@
  \end{lstlisting}
\end{frame}
\begin{frame}
  \frametitle{Frequent used commands}
  \begin{itemize}
    \item checkout(co): checkout prjoect or package
    \item commit(ci): commit local sources to server, must run in a package working directory
    \item list(ls): list project in obs or packages in a project
    \item rdelete: delete project or package on obs server
    \item deleterequest(dr): for common user who didn't have privileges, could ask for administrator to do the deletion
    \item meta: vie or edit meta info of projects or packages, to create a project \& package, just edit meta info for not existed items.
  \end{itemize}
  For more details, please see osc manpage(1) (man osc) 
\end{frame}

\begin{frame}
  \frametitle{Brief Introduction to OBS Webui}
  The operations we can do in webui:\\
  \begin{itemize}
    \item Register \& Login
    \item Monitor Building process
    \item Manage projects \& packages
    \item \alert{Deprecated} manage package source files
  \end{itemize}
\end{frame}

\begin{frame}
  \frametitle{Behind the scene: How did obs build packages}
  \begin{itemize}
    \item Create virtual root
    \item Do 'rpmbuild'
  \end{itemize}
\end{frame}

\section{Set up OBS Server}
\subsection{software setup}
\begin{frame}
  \frametitle{Setup OBS Softwares}
  \begin{itemize}
    \item obs needs backends \& two rails applications
    \item they can be installed via zypper
  \end{itemize}
\end{frame}

\subsection{Setup meego build projects}

\begin{frame}
  \frametitle{Setup projects for meego build Part1: Create Projects}
  Using oscl meta -e projectname
  Workflow:\\
  \begin{itemize}
    \item Set name and description.
    \item Add repositories and target archs.
  \end{itemize}

\end{frame}

\begin{frame} [fragile]
  \frametitle{Setup projects for meego build Part 2: Prepare resources}
  I have a script toolkit for setup resources, dowload from http://cnbjas56.cn.ebgroup.elektrobit.com/download/tools/obsmi.tar.bz2 \\
  Workflow:\\
  \begin{itemize}
    \item Install obsmi, sudo ./setup install (./setup --help see more info)
    \item Check your repos.conf, to manage resources to download.
  \end{itemize}
  Sample of repos.conf: \\
  \lstset{language=}
  \begin{lstlisting}
[daily-core-armv7l]
remote = http://repo.meego.com/MeeGo/builds/trunk/daily/core/repos/armv7l/packages
local = /mirror/daily/core/armv7l
link = meego armv7l core-armv7l
  \end{lstlisting}
\end{frame}

\begin{frame}[fragile]
  \frametitle{Setup projects for meego build Part 3: Make links}
  Now we have all packages ready, if you're using misched, links would create automatically, if not, you can use the tool 'milink' sepratedly. \\
  USAGE: milink prj repo archfile localrepo [-u][-h]
  An archfile is discribing a mapping relation from repo to obs building, here is a sample: \\

  \lstset{language=perl}
  \begin{lstlisting}
  armv7l armv7el
  noarch armv7el
  \end{lstlisting}
\end{frame}

\begin{frame}
  \frametitle{Setup projects for meego build Part 4: Using SubProjects}
  Using subproject can reuse the exsiting projects' repository, we can setup and config subproject form the web interface\\
\end{frame}

\section{OBS Maintainance}

\begin{frame}
  \frametitle{Upgrate}
  \begin{itemize}
    \item Upgrate software
    \item Check configs
    \item Update database, rake db:migrate
  \end{itemize}
  
\end{frame}
\begin{frame}
  \frametitle{Bakup OBS}
  \begin{itemize}
    \item backup database
    \item backup projects contents
    \item backup mirrors
  \end{itemize}
\end{frame}

\section{Conclusion}
\begin{frame}
  OBS is good, enjoy using it. \\
  Thank you for your attending.
\end{frame}

\end{document}

% vim:set fenc=utf8:
% vim:set ft=tex:


